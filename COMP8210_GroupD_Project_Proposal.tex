\documentclass{beamer}

\usepackage[utf8]{inputenc}
\usepackage{graphicx}

\usetheme{CambridgeUS}

%Information to be included in the title page:
\title{Using millions of emoji occurrences to learn any-domain representations for detecting sentiment, emotion and sarcasm}
\author{
\textbf{COMP8240 Group D Project Proposal} \\
\vspace{5mm} %5mm vertical space
Ajit Kumar Chaudhary (Student id:- 46124942) \\
Avinash Panday (Student id:- 46139001) \\
Saroj Bogati (Student id:- 45842620) \\
Moiz Ahmed Khan (Student id:- 45785031) \\
}

\institute{Macquaire University}
\date{2020-09-01}





\begin{document}

\frame{\titlepage}

\begin{frame}

\frametitle{Overview}

\begin{itemize}
\item Motivation 
\item Benchmark datasets 
\item Method
\item Demo
\end{itemize}

\end{frame}


\begin{frame}
\frametitle{Motivation}
\begin{columns}
\column{.5\textwidth}
It is published at  \textbf{EMNLP} and globaly ranked 1st in \textbf{Sarcasm detection}. It uses 1246 million tweets containing about 64 common emojis on 8 benchmark dataset to predict sentiment, emotion, and sarcasm. 
\column{.5\textwidth}
\includegraphics[height = .40\textheight]{emoji.png}
\end{columns}
\end{frame}

\begin{frame}

\frametitle{Benchmark datasets}
Code can be found in \url{https://paperswithcode.com/paper/using-millions-of-emoji-occurrences-to-learn}. It uses 8 benchmark datasets.
\begin{itemize}
\item \textbf{Olympic:} negative and high control, positive and high control, negative and low control, positive and high control.
\item \textbf{PsychExp:} joy, fear, anger, sadness, disgust, shame, guilt
\item \textbf{SCv1:} not sarcastic, sarcastic
\item \textbf{SCv2-GEN:} not sarcastic, sarcastic
\item \textbf{SE0714:} fear, joy, sadness
\item \textbf{SS-Twitter:} negative, positive
\item \textbf{SS-Youtube:} negative, positive
\item \textbf{kaggle-insults:} neutral, insult
\end{itemize}
\end{frame}

\begin{frame}
\frametitle{Method}
\begin{itemize}
\item \textbf{Pretraining:-} The data is split into a training, validation and test set, where the validation and test set is randomly sampled in such a way that each emoji is equally represented.
\item \textbf{Model Used:-} It uses LSTM, fast text classification, DeepMoji model to predict the accuracy of the classifiers.
\item \textbf{Transfer learning:-} It uses approach in which 
all layers in the model are frozen when fine-tuning on the target task except the last layer.
\end{itemize}
\end{frame}


\begin{frame}
\frametitle{Method}
\includegraphics[height = .40\textheight]{layers.png}
\vspace{5mm} %5mm vertical space

In the figure, each layer is fine tuned seperately. The layers covered in blue rectangle are frozen.
\end{frame}




\begin{frame}

\frametitle{Demo}

We will be reproducing the demo as per \url{https://deepmoji.mit.edu/}

\includegraphics[height = .70\textheight]{deepemoji.png}

\end{frame}

\end{document}